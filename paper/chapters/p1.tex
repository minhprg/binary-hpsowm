% Genetic Algorithms
Among metaheuristic methods, Genetic Algorithm was the first algorithm that utilizes the power of biological mechanisms, and more specifically, the power of natural selection and evolution. It was firstly introduced by Holland in the 1970s \cite{holland1992genetic}. In this algorithm, starting from an initial population of chromosomes (which are bits that compose the program), a proportion of the best of them are selected, based on a fitness function, to create a new population. Those selected chromosomes are then enter a reproduction (or recombination) process where more than one of them are used to produce a new child solution. The children are then chosen with probability for mutation process, where one or some genes in a chromosome are changed from its initial state to maintain the genetic diversity. Finally we have a new population and the process continues until we reach the optimal solution or time/memory limits.

Genetic Algorithms can provide a good approximation of the solution of all types of problems and it has been successfully applied in many problems including scheduling, optimization of networks, engineering, rule discovery \ldots \cite{ross1994applications}

Where GA was influenced by the biological mechanisms, another evolutionary algorithm called Particle Swarm Optimization (PSO) which is introduced by Kennedy and Eberhard in 1995 \cite{kennedy2010particle}, simulates the the movement of organisms in a bird flock or fish school. It was initially used as an optimization technique for use in real-number problem spaces \cite{kennedy1997discrete}. Since its creation, there have been many variants which were proposed. They mostly focus on tuning the velocity of the particle, to balance between local search and global search \cite{shi1998modified}\cite{clerc2002particle}. In 1997, Kennedy and Eberhart modified PSO into a binary search space version. They tested with De Jong's five test functions \cite{de1975analysis} and concludes that binary PSO is capable of solving these problems rapidly. 

In this research, we adapt many ideas from binary PSO implementation to propose a binary version of Hybrid PSO with Wavelet Mutation (HPSOWM), which is the work of \cite{ling2008hybrid} to solve continuous search space. When HPSOWM was proven to outweigh other approaches (HPSOM \cite{esmin2005hybrid}, HGAPSO \cite{juang2004hybrid}, HGPSO \cite{noel2004simulation}) in finding optimal solution, convergence rate \ldots, it hasn't been tested in binary version yet. That is the purpose of this research, to implement binary version of HPSOWM, to evaluate its performance against GA and binary PSO (BSPO) using many different benchmark test functions which are divided into three categories of different characteristics.

This paper is organized as follows: the first section gives the introduction. The second section discusses the backgrounds of this research which are PSO and HPSOWM. Next we discuss about how to transform those two algorithms from continuous to binary version. After that, we describe the experiment with settings, benchmark functions and analyze the results. Finally a conclusion is given with the overall result of this proposed model.